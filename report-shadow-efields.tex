\documentclass[a4paper,10pt]{article}
\usepackage[utf8]{inputenc}
\usepackage[]{graphicx}
\usepackage{rotating}
\usepackage{hyperref}

\usepackage{color}
\usepackage{listings}

% \lstset{basicstyle=\small,
% % keywordstyle=\bf \color{blue},
% % identifierstyle=\underline,
% % commentstyle=\color[red]{0.5},
% stringstyle=\ttfamily \color{red},
% showstringspaces=false}
\lstset{language=Fortran,
        keywordstyle=\color{blue}\textbf,
	commentstyle=\color[rgb]{0.133,0.545,0.133},
	stringstyle=\color[rgb]{0.627,0.126,0.941},
	breaklines=true,
	showstringspaces=false,
	frame=trBL, basicstyle=\scriptsize %\small %\tiny %\footnotesize
       }


%opening
\title{How SHADOW manipulates electric vectors}
\author{M. Sanchez del Rio}

\begin{document}
%\lstset{language=[95]Fortran}
\maketitle

\begin{abstract}
This document explains how the electric fields and related parameters (intensity, polarization) are
defined and modified in SHADOW. It also describes a bug fixed in March 2013. 
\end{abstract}

\section{Definitions} 

Most Monte-Carlo codes store a parameter ``intensity'' or weight for each ray that allow to track the energy or 
intensity of each traced event. As most of electromagnetic interactions are better described by the change in
the electric fieldit is more appropriated
to store directly the electric fields and not the intensity. The intensity is the squarred-modulus of 
the electric fields, The electric fields permit not only to compute
intensity or each ray, but also the polarization, phase changes, and coherence properties. In order to extract
usable and accurate information on these important parameters, the electric fields must be correctly simulated and
propagated. I try to summarize here some of the assumptions and manipulations of the electric fields in SHADOW. 

A single ray $\vec{R}$ in SHADOW is an array of 18 parameters (called ``columns'' in SHADOW's jargon) stored
in the following order:

\begin{equation}
   \vec{R}=( \vec{x},\vec{v},\vec{E}_\sigma,f,k,n,opd,\phi_\sigma,\phi_\pi,\vec{E}_\pi)
\end{equation}
where $\vec{x}$ are the spatial coordinates, $\vec{v}$ the director cosines (ray direction), $f$ is a flag (good or
lost ray), $k$ is the wavevector modulus in $cm^{-1}$ ($k=2\pi / \lambda$, with $\lambda$ the photon wavelength ),
$n$ is a counter (from one to the total number of rays), $opd$ is the optical path (the travelled
distance times the refraction index). The complex electric vector is expressed as a function of the parallel
($\sigma$) and perpendicular ($\pi$) components:

\begin{equation}
   \vec{E}= \vec{E}_\sigma e^{i \phi_\sigma} + \vec{E}_\pi e^{i \phi_\pi}
\end{equation}.

Obviously the storage in SHADOW is redundant, for example $v_3=\sqrt{v_1^2+v_2^2}$ and regarding the electric
fields:
\begin{itemize}
 \item $\vec{E}_\sigma \perp \vec{E}_\pi$, $\vec{E}_\sigma \perp \vec{v}$, $\vec{E}_\pi \perp \vec{v}$
 \item $\vec{E}_\sigma$ and $\vec{E}_\pi$ are real.
 \item $|\vec{E}|^2=1$ at the source, and $|\vec{E}|$ is reduced as far as the ray interacts with the different 
 optical elements.
\end{itemize}

The intensity of the ray is:
\begin{eqnarray}
   I=|\vec{E}|^2 = \vec{E}.\vec{E}^* = (\vec{E}_\sigma e^{i \phi_\sigma} + \vec{E}_\pi e^{i\phi_\pi} ).
   (\vec{E}_\sigma e^{-i\phi_\sigma} + \vec{E}_\pi e^{-i\phi_\pi} ) =  \nonumber \\
   |\vec{E}_\sigma|^2 + |\vec{E}_\pi|^2 + 2 \vec{E}_\sigma.\vec{E}_\pi \cos(\phi_\sigma-\phi_\pi) =
   |\vec{E}_\sigma|^2 + |\vec{E}_\pi|^2
\end{eqnarray}.


\section{A polarized source}

When simulating a source, SHADOW first samples the ray energy, direction and positions and then define set electric vectors. 
A first step is to obtain orthogonal unitary vectors $\vec{u}_\sigma$ and $\vec{u}_\pi$ normal to $\vec{v}$. At the source 
level, there is not yet knowledge of any optical surface to define $\sigma$ and $\pi$ directions, so these directions 
are considered ``mostly'' horizontal and vertical, respectively. The unitary vectors are calculated as: 
\begin{eqnarray}
   \vec{u}_\sigma = \vec{v} \wedge (\vec{u}_1 \wedge{v}) \nonumber \\
   \vec{u}_\pi = \vec{u}_\sigma \wedge  {v}
\end{eqnarray}
where $\vec{u}_1=(1,0,0)$ the unitary vector along the horizontal direction at the source plane. 
This is done with the following code in the {\tt sourceGeom} subroutine of the 
{\tt shadow\_kernel.F90} module: 

\begin{lstlisting}
    A_VEC(1)		=   1.0D0
    A_VEC(2)		=   0.0D0
    A_VEC(3)		=   0.0D0
    ! C
    ! C   Rotate A_VEC so that it will be perpendicular to DIREC and with the
    ! C   right components on the plane.
    ! C 
    CALL	CROSS	(A_VEC,DIREC,A_TEMP)
    CALL	CROSS	(DIREC,A_TEMP,A_VEC)
    CALL	NORM	(A_VEC,A_VEC)
    CALL	CROSS	(A_VEC,DIREC,AP_VEC)
    CALL	NORM	(AP_VEC,AP_VEC)
\end{lstlisting} 


Then the electric vectors are these versors times the corresponding ($\sigma$ or $\pi$) modulus, in such a way that 
the intensity is one. The mudulus of each component is a function of a ``particular'' polarization degree defined as:
\begin{equation}
   P = \frac{\cos \chi}{\cos \chi + \sin \chi}
\end{equation}
where $\chi$ is the angle between the $\vec{E}$ vector and $\vec{u}_1$ when the beam is linearly polarized. 
The electric vector components are: 
\begin{eqnarray}
   \vec{E}_\sigma = \frac{P}{\sqrt{1-2 P+ 2 P^2}} \vec{u}_\sigma \nonumber \\
   \vec{E}_\pi    = \frac{1-P}{\sqrt{1-2 P+ 2 P^2}} \vec{u}_\pi.
\end{eqnarray}


The phase $\phi_\sigma$ is set to zero for coherent beams or to a random angle in $[0,180^{\circ})$ for an incoherent beam, and
$\phi_\pi=\phi_\sigma+\Phi$ with: 
\begin{itemize}
 \item $\Phi=0$ for linearly polarizad light 
 \item $\Phi=90^{\circ}$ for right (CW) elliptical polarization
 \item $\Phi=-90^{\circ}$ for left (CCW) elliptical polarization.
\end{itemize}
A few examples: 
\begin{itemize}
 \item $P=1$,$\Phi=0$ for horizontal linearly polarizad light 
 \item $P=0$,$\Phi=0$ for vertical linearly polarizad light
 \item $P=0.5$,$\Phi=0$ for $45^\circ$ linearly polarizad light
 \item $P=0.5$ $\Phi=90^{\circ}$ for right circular polarization
 \item $P<0.5$,$\Phi=-90^{\circ}$ for left elliptical polarization mostly horizontal
\end{itemize}


\section{Change of reference frame}

The beam created in the source reference frame must be converted into the optical element reference frame for performing the
ray tracing. The rotation of the vectors from the source (or {\it previous} o.e. to the o.e. under study is done 
in the subroutine {\tt RESTART}. It applies sequential rotations of the vectors using the mirror orientation angle $\alpha$,
incident angle $\theta_i$, etc. An example of code is the following, for changing the $\vec{E}_\pi$:

\begin{lstlisting}
 
TEMP_1P(1) =   AP(1,ITIME)*COSAL + AP(3,ITIME)*SINAL
TEMP_1P(2) =   AP(2,ITIME)
TEMP_1P(3) = - AP(1,ITIME)*SINAL + AP(3,ITIME)*COSAL

TEMP_2P(1) =   TEMP_1P(1)
TEMP_2P(2) =   TEMP_1P(2)*SINTHR + TEMP_1P(3)*COSTHR
TEMP_2P(3) = - TEMP_1P(2)*COSTHR + TEMP_1P(3)*SINTHR

TEMP_1P(1) =   TEMP_2P(1)*COSAL_S - TEMP_2P(2)*SINAL_S
TEMP_1P(2) =   TEMP_2P(1)*SINAL_S + TEMP_2P(2)*COSAL_S
TEMP_1P(3) =   TEMP_2P(3)

AP(1,ITIME)	= TEMP_1P(1)
AP(2,ITIME)	= TEMP_1P(2)
AP(3,ITIME)	= TEMP_1P(3)
\end{lstlisting}

Then the subroutine {\tt MIRROR1} performs the reflection/refraction/diffraction by the optical element (o.e.), 
changing the beam direction and electric vectors. This will be discussed in detail in the next paragraphs.
Last, after the reflection, the rays are projected into the image plane. For that it projects the vectors 
vectors onto the versors of the new reference. For example, for $\vec{E}_\pi$

\begin{lstlisting}
AP_VEC(1) = AP(1,J)
AP_VEC(2) = AP(2,J)
AP_VEC(3) = AP(3,J)
CALL DOT (AP_VEC,UXIM,A_1)
CALL DOT (AP_VEC,VNIMAG,A_2)
CALL DOT (AP_VEC,VZIM,A_3)
AP(1,J) = A_1 * rr_reflectivity
AP(2,J) = A_2 * rr_reflectivity
AP(3,J) = A_3 * rr_reflectivity
\end{lstlisting}

It is important that these changes of refecence system (before and after interaction with the o.e.) 
are not affecting the phase $\phi_\sigma$ and $\phi_\pi$ and conserve the modulus of each electric field
component $\vec{E}_\sigma$ and $\vec{E}_\pi$. Also, they do not alter the orthogonality relationships:
\begin{eqnarray}
\label{ortho}
\vec{E}_\sigma \perp \vec{E}_\pi \nonumber \\
\vec{E}_\sigma \perp \vec{v} \nonumber \\
\vec{E}_\pi \perp \vec{v}
\end{eqnarray}

\section{Modifications of electric fields in ``local'' o.e.}

The physics of the X-ray reflection and diffraction in an optical surface affects in a different way 
the {\it local} parallel ($\sigma$) and perpendicular ($\pi$) components. The ``local'' parallel component
is a vector that sits on the o.e. surface and it is not coincident with the incident $\vec{E}_\sigma$.
Therefore, one must calculate the ``local'' $\vec{u}'_\sigma$ and $\vec{u}'_\pi$ vectors that verify
1) they are orthogonal to $\vec{v}$ and ii) $\vec{u}'_\sigma$ is contained in the o.e. surface. 
This is done here ( {\tt MIRROR1} subroutine):

\begin{lstlisting}
! * Check for reflectivity. If this mode is "on", we have to compute
! * some angles, namely the sine of the incidence angle and the sine
! * of the A vector with the normal. Also, the polarized light is 
! * treated as a superposition of two orthogonal A vectors with the appropriate
! * phase relation. These two incoming vectors have to be resolved into the
! * local S- and P- component with a new phase relation.
! * A_VEC will be rotated later, once the amplitude will have been determined.
     	CALL	CROSS_M_FLAG 	(VVIN,VNOR,AS_TEMP,M_FLAG)	! vector pp. to inc.pl.
!print *,'M_FLAG: ',M_FLAG

     	IF (M_FLAG.EQ.1) THEN
	CALL	DOT	(AS_VEC,AS_VEC,AS2)
	CALL	DOT	(AP_VEC,AP_VEC,AP2)
	IF (AS2.NE.0)	THEN
     	 DO 499 I=1,3
 499   	   AS_TEMP(I) = AS_VEC(I)
	ELSE
	 DO 599 I=1,3
 599	   AS_TEMP(I) = AP_VEC(I)
	END IF
     	END IF


     	CALL	NORM  	(AS_TEMP,AS_TEMP)	! Local unit As vector
	CALL	CROSS	(AS_TEMP,VVIN,AP_TEMP)
	CALL	NORM	(AP_TEMP,AP_TEMP)	! Local unit Ap vector
\end{lstlisting}

The electric vector can thus be expressed in two orthonormal vectors $\vec{E'}_\sigma$ and $\vec{E'}_\pi$ along these 
new directions $\vec{u'}_\sigma$ and $\vec{u'}_\pi$. Physically, it is a rotation of the old electric vectors around
the $\vec{v}$ direction to put the $\sigma$ component on top of the surface, but the rotation affect not only the 
moduli, but also the phases. The new (complex) electric vectors are build from the projection of the old ones onto the
new axes, thus originating a transformation in both moduli and phases:

\begin{eqnarray}
\vec{E'}_\sigma e^{i \phi'_\sigma} = [(\vec{E}_\sigma e^{i \phi_\sigma}).\vec{u'}_\sigma + (\vec{E}_\pi e^{i \phi_\pi}).\vec{u'}_\sigma ] 
  ~~\vec{u'}_\sigma  \\ 
\vec{E'}_\pi e^{i \phi'_\pi} = [(\vec{E}_\sigma e^{i \phi_\sigma}).\vec{u'}_\pi + (\vec{E}_\pi e^{i \phi_\pi}).\vec{u'}_\pi ] 
  ~~\vec{u'}_\pi 
\end{eqnarray}


Defining the projection of the (real) electric field components onto these versors as: 
\begin{eqnarray}
a_{11} = \vec{E}_\sigma . \vec{u'}_\sigma, ~~~~ 
a_{12} = \vec{E}_\sigma . \vec{u'}_\pi \nonumber \\
a_{21} = \vec{E}_\pi . \vec{u'}_\sigma, ~~~~
a_{22} = \vec{E}_\pi . \vec{u'}_\pi 
\end{eqnarray}

we get:
\begin{eqnarray}
\label{withphases}
\vec{E'}_\sigma e^{i \phi'_\sigma} = (a_{11} e^{i \phi_\sigma} + a_{12} e^{i \phi_\pi}) \vec{u'}_\sigma \nonumber \\ 
\vec{E'}_\pi e^{i \phi'_\pi} =    (a_{21} e^{i \phi_\sigma} + a_{22} e^{i \phi_\pi}) \vec{u'}_\pi
\end{eqnarray}

And the moduli are: 

\begin{eqnarray}
|\vec{E'}_\sigma|^2 = \vec{E'}_\sigma . \vec{E'}_\sigma^*  = 
a_{11}^2 + a_{12}^2 + 2 a_{11} a_{12} \cos(\phi_\sigma-\phi_\pi) \equiv
M_\sigma^2 \nonumber \\ 
|\vec{E'}_\pi|^2    = \vec{E'}_\pi    . \vec{E'}_\pi^*     = 
a_{21}^2 + a_{22}^2 + 2 a_{21} a_{22} \cos(\phi_\sigma-\phi_\pi) \equiv
M_\pi^2
\end{eqnarray}

Therefore we can construct the new local (real) electric fields as: 
\begin{eqnarray}
\label{final1}
\vec{E'}_\sigma = M_\sigma \vec{u'}_\sigma  \nonumber \\ 
\vec{E'}_\pi = M_\pi \vec{u'}_\pi  
\end{eqnarray}

To compute the new phases, we insert Eq.~\ref{final1} in Eq.~\ref{withphases} and get: 
\begin{eqnarray}
e^{i \phi'_\sigma} = M_\sigma^{-1} (a_{11} e^{i \phi_\sigma} + a_{12} e^{i \phi_\pi})  \nonumber \\ 
e^{i \phi'_\pi} =  M_\pi^{-1}      (a_{21} e^{i \phi_\sigma} + a_{22} e^{i \phi_\pi}) 
\end{eqnarray}

Therefore:
\begin{eqnarray}
\label{final2}
\tan{\phi'_\sigma} = \frac{Im(a_{11} e^{i \phi_\sigma} + a_{12} e^{i \phi_\pi})}
                          {Re(a_{11} e^{i \phi_\sigma} + a_{12} e^{i \phi_\pi})} = 
                          \frac{a_{11} \sin{\phi_\sigma} + a_{12} \sin{\phi_\pi}}
                               {a_{11} \cos{\phi_\sigma} + a_{12} \cos{\phi_\pi}}  \nonumber \\ 
\tan{\phi'_\pi} =  \frac{Im(a_{21} e^{i \phi_\sigma} + a_{22} e^{i \phi_\pi})}
                        {Re(a_{21} e^{i \phi_\sigma} + a_{22} e^{i \phi_\pi})} = 
                        \frac{a_{21} \sin{\phi_\sigma} + a_{22} \sin{\phi_\pi}}
                             {a_{21} \cos{\phi_\sigma} + a_{22} \cos{\phi_\pi}}
\end{eqnarray}


The code that implements Eq.~\ref{final1} and Eq.~\ref{final2} is in {\tt MIRROR1}: 

\begin{lstlisting}
CALL	DOT	(AS_VEC,AS_TEMP,A11)
CALL	DOT	(AP_VEC,AS_TEMP,A12)
CALL	DOT	(AS_VEC,AP_TEMP,A21)
CALL	DOT	(AP_VEC,AP_TEMP,A22)
! ** Now recompute the ampltitude and phase of the local S- and P- component.
AS_NEW	= SQRT(ABS(A11**2 + A12**2 + 2.0D0*A11*A12*COS(PHS-PHP)))
AP_NEW	= SQRT(ABS(A21**2 + A22**2 + 2.0D0*A21*A22*COS(PHS-PHP)))
CALL DOT(AP_VEC,AS_VEC,TMP11)

CALL	SCALAR	(AS_TEMP,AS_NEW,AS_VEC)	! Local As vector
CALL	SCALAR	(AP_TEMP,AP_NEW,AP_VEC)	! Local Ap vector

PHTS	= A11*SIN(PHS) + A12*SIN(PHP)
PHBS	= A11*COS(PHS) + A12*COS(PHP)
PHTP	= A21*SIN(PHS) + A22*SIN(PHP)
PHBP	= A21*COS(PHS) + A22*COS(PHP)
CALL	ATAN_2	(PHTS,PHBS,PHS)		! Phase of local As vector
CALL	ATAN_2	(PHTP,PHBP,PHP)		! Phase of local Ap vector
\end{lstlisting}



Note that these transformations imply a polarization mixing, or not conservation of the 
intensity of each component during the projection ($|\vec{E'}_\sigma| \ne |\vec{E}_\sigma|$),
but the total intensity is conserved ($I=|\vec{E}_\sigma|^2+|\vec{E}_\pi|^2=
|\vec{E'}_\sigma|^2+|\vec{E'}_\pi|^2$). The phases also change, but the difference of thase 
is conserved $\Phi=\phi_\sigma-\pi_\pi=\phi'_\sigma-\phi'_\pi$ (I think, but I have not demostrated it).
The new electric vectors are orthogonal to $\vec{v}$ by constructions.

\section{Reflection/refraction in the o.e. and subsequent changes in electric vectors}
Once the electric vectors are expressed in the o.e. local coordinates ($\vec{E'}_\sigma$ and 
$\vec{E'}_\pi$) they are multiplied by the mirror/crystal reflectivities and the phases also affected: : 
\begin{eqnarray}
 \label{reflectivities}
 \vec{E'}_\sigma^{new} = \vec{E'}_\sigma R_\sigma \nonumber \\
 \vec{E'}_\pi^{new} = \vec{E'}_\pi R_\pi \nonumber \\
 {\phi '}_\sigma^{new} = {\phi '}_\sigma \Sigma_\sigma \nonumber \\
 {\phi '}_\pi^{new} = {\phi '}_\pi \Sigma_\pi 
\end{eqnarray}
where $R_\sigma$ and $R_\pi$ are the o.e. reflectivities and $\Sigma_\sigma$ and $\Sigma_\pi$ are the 
phases added in the reflection/refraction/diffraction. This process is accompanied by a change of 
direction of the ray from the incident direction $\vec{v}$ to a new output direction $\vec{v'}$. 
All these operations are done in {\tt MIRROR1} subroutine.

The resulting electric vectors and phases after applying Eq.~\ref{reflectivities} must now hold
the orthogonality relations (Eq.~\ref{ortho}) with respect to $\vec{v'}$ so they must be changed. 
\color{red}
This part was incorrectly
done in the following code, which assumes conservation of the $\vec{E'}_\sigma$ component (except for gratings)
and the $\pi$ component is ``mirrored''. These operations do not garantee that the resulting 
vectors are normal to $\vec{v'}$. In some cases, in particular for $\sigma$ polarizad light onto 
o.e.'s with mirror orientation angle $90^\circ$ can result in a loss of beam intensity in 
subsequent changes of frames. The {\bf wrong} code was (in {\tt MIRROR1}): 
\color{black}

\begin{lstlisting}
! ** So far we have the new amplitude of the two components. We have now
! ** to 'reflect' A_VEC onto the mirror. For this, notice that the s-comp
! ** is geometrically unchanged, while the p-comp is changed. The angles
! ** are exchanged with respect to VVIN. Things are more complicated in
! ** the case of a grating, due to the vectorial nature of the diffraction,
! ** not treated here. We make the simplifying assumption that the
! ** diffraction will not change the degree of polarization. This mean that
! ** A_VEC will have the same components referred to the ray as before the
! ** diffraction. 

VVOUT(1)	=   RAY(4,ITIK)
VVOUT(2)	=   RAY(5,ITIK)
VVOUT(3)	=   RAY(6,ITIK)
! C 
! C The following IF block applies only to the GRATING case.
! C The binormal is redefined in terms of the diffraction
! C plane.
! C
     	IF (F_GRATING.NE.0.OR.F_BRAGG_A.EQ.1) THEN
	  CALL	PROJ	(VVOUT,VNOR,VTEMP)
	  CALL	SCALAR	(VTEMP,-2.0D0,VTEMP)
	  CALL	SUM	(VTEMP,VVOUT,VTEMP)
	  !CALL	CROSS	(VTEMP,VNOR,AS_TEMP)
	  CALL	CROSS_M_FLAG	(VTEMP,VNOR,AS_TEMP,M_FLAG)
     	 IF (M_FLAG.EQ.1) THEN
	   CALL	DOT	(AS_VEC,AS_VEC,AS2)
	   CALL	DOT	(AP_VEC,AP_VEC,AP2)
	  IF (AS2.NE.0)	THEN
     	    DO 899 I=1,3
 899          AS_TEMP(I) = AS_VEC(I)
	  ELSE
	   DO 999 I=1,3
 999	     AS_TEMP(I) = AP_TEMP(I)
     	  END IF
	 END IF
     	  CALL	NORM  	(AS_TEMP,AS_TEMP)	! Local unit As vector
	  CALL	CROSS	(AS_TEMP,VTEMP,AP_TEMP)
	  CALL	NORM	(AP_TEMP,AP_TEMP)	! Local unit Ap vector
     	  CALL	DOT	(AS_VEC,AS_VEC,RES)
     	  RES	=    SQRT (RES)

     	  CALL	SCALAR	(AS_TEMP,RES,AS_VEC)
	  CALL	DOT	(AP_VEC,AP_VEC,RES)
	  RES	=    SQRT (RES)
	  CALL	SCALAR	(AP_TEMP,RES,AP_VEC)
     	END IF


CALL	PROJ	(AP_VEC,VNOR,VTEMP)
CALL	VECTOR	(VTEMP,AP_VEC,VTEMP)
CALL	SCALAR	(VTEMP,-2.0D0,VTEMP)

CALL	SUM	(AP_VEC,VTEMP,AP_VEC)
\end{lstlisting}

\color{red}
This code has been replaced by a new one that simply calculates the new $\sigma$ and $\pi$ versors orthogonal to 
$\vec{v'}$ and affects them by the electric field moduli that do not change. Also, the phases are not changed. 
This guaratees that the intensity is conserved, as well as orthogonality. The new code is: 
\color{black}

\begin{lstlisting}
! Electric vectors are changed to assure orthogonality with the new direction VVOUT
! To conserve intensity, the moduli of Es and Ep must not change
! AS_VEC and VVOUT are not orthogonal so a projection of S and P coordinates into the
! new ones do not work as it may be a component of the electric field along VVOUT

CALL CROSS_M_FLAG  (VVOUT,VNOR,AS_TEMP,M_FLAG) ! vector pp. to inc.pl.
CALL DOT (AS_VEC,AS_VEC,AS2)
CALL DOT (AP_VEC,AP_VEC,AP2)

IF (M_FLAG.EQ.1) THEN
 IF (AS2.NE.0) THEN
   DO I=1,3
     AS_TEMP(I) = AS_VEC(I)
   END DO
 ELSE
  DO I=1,3
   AS_TEMP(I) = AP_VEC(I)
  END DO
 END IF
END IF

CALL NORM   (AS_TEMP,AS_TEMP) ! Local unit As vector perp to vvout
CALL CROSS (AS_TEMP,VVOUT,AP_TEMP)
CALL NORM (AP_TEMP,AP_TEMP) ! Local unit Ap vector perp to vvout

do i=1,3
  as_vec(i) = as_temp(i) * sqrt(as2)
  ap_vec(i) = ap_temp(i) * sqrt(ap2)
end do
\end{lstlisting}




\end{document}
